%
%------------------------------------------------------------------------------------
%	AUTORI: Morati Mirko, Noè Murr
%------------------------------------------------------------------------------------
%

\documentclass[a4paper,11pt]{article}
\usepackage[T1]{fontenc}
\usepackage[utf8x]{inputenc}
\usepackage[italian]{babel}
\usepackage{amsmath}
\usepackage{calc}
\usepackage[a4paper]{geometry}
\usepackage[usenames,dvipsnames]{xcolor}
\usepackage{fancyhdr} % Required for custom headers
\usepackage{lastpage} % Required to determine the last page for the footer
\usepackage{extramarks} % Required for headers and footers
%\usepackage[usenames,dvipsnames]{color} % Required for custom colors
\usepackage{graphicx} % Required to insert images
\usepackage{listings} % Required for insertion of code
\usepackage{courier} % Required for the courier font
\usepackage{lipsum} % Used for inserting dummy 'Lorem ipsum' text into the template
\usepackage{pdfpages}
\usepackage{listings}
\usepackage{mdframed}
\usepackage{enumitem}
%\usepackage{minted}
\usepackage{float}
\usepackage{array}
\usepackage{zref-xr}
\zexternaldocument[e-]{elaborato}[elaborato.pdf]

%\renewcommand\listingscaption{}
%\newcommand\ceil[1]{\lceil#1\rceil}
\usepackage{zi4}
\usepackage{tikz}
\usetikzlibrary{automata, arrows, positioning, calc, matrix, shapes.geometric, chains}
\usepackage{verbatim}

\usepackage{color}   %May be necessary if you want to color links
\usepackage{hyperref}
\hypersetup{
	colorlinks,
	citecolor=black,
	filecolor=black,
	linkcolor=black,
	urlcolor=black
}



\newenvironment{framescelte}[1]
{\mdfsetup{
		frametitle={\colorbox{white}{\space#1\space}},
		innertopmargin=6pt,
		frametitleaboveskip=-\ht\strutbox,
		frametitlealignment=\center
	}
	\begin{mdframed}%
	}
	{\end{mdframed}}


% Margins
\topmargin=-0.45in
\evensidemargin=0in
\oddsidemargin=0in
\textwidth=6.25in
\textheight=9in
\headsep=0.25in

\linespread{1.1} % Line spacing


% Set up the header and footer
\pagestyle{fancy}
%\lhead{\hmwkAuthorName} % Top left header
\chead{\hmwkTitle} % Top center head
\rhead{\firstxmark} % Top right header
\lfoot{\lastxmark} % Bottom left footer
%\cfoot{} % Bottom center footer
%\rfoot{Page\ \thepage\ of\ \protect\pageref{LastPage}} % Bottom right footer
\renewcommand\headrulewidth{0.4pt} % Size of the header rule
\renewcommand\footrulewidth{0.4pt} % Size of the footer rule

%------------------------------------------------------------------------------------
%	TITOLO
%------------------------------------------------------------------------------------
%----------------------------------------------------------------------------------------
%	NAME AND CLASS SECTION
%----------------------------------------------------------------------------------------

\newcommand{\hmwkTitle}{Codice\ Elaborato\ Assembly}
\newcommand{\hmwkClass}{Architettura degli Elaboratori}
\newcommand{\hmwkAuthorName}{Mirko Morati,\ Noè Murr}

%------------------------------------------------------------------------------------
%	TITLE PAGE
%------------------------------------------------------------------------------------

\title{
	\vspace{2in}
	\textmd{\textbf{\hmwkClass:\\ \hmwkTitle}}\\
	\vspace{0.1in}\large{\textit{\hmwkAuthorName}}
	\vspace{3in}
}

%\author{}

%------------------------------------------------------------------------------------


%------------------------------------------------------------------------------------
%	FORMATTAZIONE ELEMENTI
%------------------------------------------------------------------------------------
%\renewcommand*{\ttdefault}{zi4}
%\newcommand{\stato}[1]{\textbf{\fontfamily{zi4}\selectfont #1}}
%\newcommand{\signin}[1]{\textcolor{BrickRed}{\fontfamily{zi4}\selectfont #1}}
%\newcommand{\signout}[1]{\textcolor{Blue}{\fontfamily{zi4}\selectfont #1}}
%\newcommand{\inctxt}[1]{\textit{\fontfamily{zi4}\selectfont #1}}
%------------------------------------------------------------------------------------

\renewcommand{\labelitemi}{$\cdot$}

\newcommand{\Assembly}{\texttt{Assembly} }

\newcommand{\itemtt}[1]{\item \texttt{#1}}



\lstdefinestyle{BashStyle}{
	language=bash,
	basicstyle=\ttfamily,
	numbers=left,
	numberstyle=\tiny\ttfamily\color{black},
	numbersep=-10pt,
	frame=tb,
	columns=fullflexible,
	title=\textit{},
	emph={souce, ps, map, -s, rl, rlib},emphstyle={\bfseries}
}

\definecolor{mygreen}{rgb}{0,0.6,0}
\definecolor{mygray}{rgb}{0.5,0.5,0.5}
\definecolor{mymauve}{rgb}{0.58,0,0.82}


\lstdefinelanguage{MyAssembler}{
	morecomment=[l][\color{mygray}]{\#},
	morekeywords=[2]{eax, ebx, ecx, edx, edi, esp, ebp, al, bl, cl, dl, ah, bh, ch, dh},
%	morekeywords=[3]{\$, 0, 1, 2, 3, 4, 5, 6, 7, 8, 9},
	morekeywords=[4]{popl,popb,pushl,pushb,jne,jn,call,int,movl,movb,xorl,xorb,subl,subb,addl,addb,mull,mulb,incl,cmpb,cmpl,jl,jg,jge,jle,je,jmp,ret},
	morekeywords=[5]{section,data,type,long,bss,text,globl,equ,ascii,asciz,include},
	morestring=[b][\color{mymauve}]{"},
	morekeywords=[6][\color{Violet}]{},
	morekeywords=[7]{input\_fd,output\_fd,init,reset,rpm,alm,numb,mod}
	}
	
\lstdefinestyle{MyAsm}{ %
	backgroundcolor=\color{white},   % choose the background color; you must add \usepackage{color} or \usepackage{xcolor}
	basicstyle=\footnotesize\ttfamily,        % the size of the fonts that are used for the code
	breakatwhitespace=false,         % sets if automatic breaks should only happen at whitespace
	breaklines=true,                 % sets automatic line breaking
	captionpos=b,                    % sets the caption-position to bottom
	commentstyle=\color{mygray},    % comment style
	deletekeywords={...},            % if you want to delete keywords from the given language
	escapeinside={\%*}{*)},          % if you want to add LaTeX within your code
	extendedchars=true,              % lets you use non-ASCII characters; for 8-bits encodings only, does not work with UTF-8
	literate={è}{{\'e}}1 {ì}{{\'i}}1 {é}{{\'e}}1  {ò}{{\'o}}1 {à}{{\'a}}1,
	frame=single,	                   % adds a frame around the code
	keepspaces=true,                 % keeps spaces in text, useful for keeping indentation of code (possibly needs columns=flexible)
	keywordstyle=\color{blue},       % keyword style
	keywordstyle=[2]\color{BrickRed},
%	keywordstyle=[3]\color{green},
	keywordstyle=[4]\color{blue},	
	keywordstyle=[5]\color{Green},
	keywordstyle=[7]\color{Orange},
	%language={[x86masm]Assembler},                 % the language of the code
	otherkeywords={...},           % if you want to add more keywords to the set
	numbers=left,                    % where to put the line-numbers; possible values are (none, left, right)
	numbersep=5pt,                   % how far the line-numbers are from the code
	numberstyle=\tiny\color{mygray}, % the style that is used for the line-numbers
	rulecolor=\color{black},         % if not set, the frame-color may be changed on line-breaks within not-black text (e.g. comments (green here))
	showspaces=false,                % show spaces everywhere adding particular underscores; it overrides 'showstringspaces'
	showstringspaces=false,          % underline spaces within strings only
	showtabs=false,                  % show tabs within strings adding particular underscores
	stepnumber=1,                    % the step between two line-numbers. If it's 1, each line will be numbered
	stringstyle=\color{mymauve},     % string literal style
	tabsize=4,	                   % sets default tabsize to 2 spaces
	%title=\lstname                   % show the filename of files included with \lstinputlisting; also try caption instead of title
}



\begin{document}
	\clearpage
	\maketitle
	\thispagestyle{empty}
	\newpage
	\tableofcontents
	\newpage
	
	\section{main.s}
%	\lstinputlisting[style=MyAsm, language = MyAssembler]{../src/main.s}
	\begin{lstlisting}[language=MyAssembler, style=MyAsm]
	# Progetto Assembly 2016
	# File: main.s
	# Autori: Noè Murr, Mirko Morati
	#
	# Descrizione: File principale, punto di inizio del programma.
	.include    "syscall.inc"
	
	.section    .data
	input_fd:   .long 0         # variabile globale che conterrà il file
	# descriptor del file di input
	
	output_fd:  .long 0         # variabile globale che conterrà il file
	# descriptor del file di output
	
	# Variabili globali per i segnali di input
	init:   .long 0
	reset:  .long 0
	rpm:    .long 0
	
	# Variabili globali per i segnali di output
	alm:    .long 0
	numb:   .long 0
	mod:    .long 0
	
	# Codice del programma
	
	.section    .text
	.globl  input_fd
	.globl  output_fd
	.globl  init
	.globl  reset
	.globl  rpm
	.globl  alm
	.globl  numb
	.globl  mod
	.globl  _start
	
	# Stringa per mostrare l'utilizzo del programma in caso di parametri errati
	usage:  .asciz "usage: programName inputFilePath outputFilePath\n"
	.equ    USAGE_LENGTH, .-usage
	
	_start:
	# Recupero i parametri del main
	popl    %eax                # Numero parametri
	
	# Controllo argomenti, se sbagliati mostro l'utilizzo corretto
	cmpl    $3, %eax
	jne     _show_usage
	
	popl    %eax                # Nome programma
	popl    %eax                # Primo parametro (nome file di input)
	popl    %ebx                # Secondo parametro (nome file di output)
	
	# NB: non salvo ebp in quanto non ha alcuna utilità
	# nella funzione start che comunque non ritorna
	
	movl    %esp, %ebp
	
	call    _open_files         # Apertura dei file
	
	_main_loop:
	
	call    _read_line          # Leggiamo la riga
	
	cmpl    $-1, %ebx           # EOF se ebx == -1
	je      _end
	
	call    _check              # Controllo delle variabili
	
	call    _write_line         # Scrittura delle variabili di output su file
	
	jmp     _main_loop          # Leggi un altra riga finché non è EOF
	
	_end:
	
	call    _close_files        # Chiudi correttamente i file
	
	# sys_exit(0);
	movl    $SYS_EXIT, %eax
	movl    $0, %ebx
	int     $SYSCALL
	
	_show_usage:
	# esce in caso di errore con codice 1
	# sys_write(stdout, usage, USAGE_LENGTH);
	movl    $SYS_WRITE, %eax
	movl    $STDOUT, %ebx
	movl    $usage, %ecx
	movl    $USAGE_LENGTH, %edx
	int     $SYSCALL
	
	# sys_exit(1);
	movl    $SYS_EXIT, %eax
	movl    $1, %ebx
	int     $SYSCALL
	
	\end{lstlisting}
	
	\section{open\_files.s}
	\lstinputlisting[style=MyAsm, language = MyAssembler]{../src/open_files.s}
	
	\section{read\_line.s}
	\lstinputlisting[style=MyAsm, language = MyAssembler]{../src/read_line.s}
	
	\section{atoi.s}
	\lstinputlisting[style=MyAsm, language = MyAssembler]{../src/atoi.s}
	
	\section{check.s}
	\lstinputlisting[style=MyAsm, language = MyAssembler]{../src/check.s}
	
	\section{itoa.s}
	\lstinputlisting[style=MyAsm, language = MyAssembler]{../src/itoa.s}
	
	\section{write\_line.s}
	\lstinputlisting[style=MyAsm, language = MyAssembler]{../src/write_line.s}
	
	\section{close\_files.s}
	\lstinputlisting[style=MyAsm, language = MyAssembler]{../src/close_files.s}

\end{document}